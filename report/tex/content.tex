% !TeX spellcheck = fr
% !TeX encoding = UTF-8

% -- Exercice 1
\section{Partie 1 : Modélisation du problème}

Dans cette partie, nous nous intéressons à la modélisation du problème du \textit{Monde des Blocs} en tant que problème de planification.

\subsection{Définition et déclaration des variables}
Voici une liste des variables utilisées, avec leur domaine et leur signification :

\begin{itemize}
    \item $n$ : le nombre de blocs
    \item $k$ : le nombre de piles
    \item $E$ : un état donné du Monde des Blocs
    \item $actions(E)$ : l'ensemble de toutes les actions possibles pour l'état $E$
    \item $t(E, i \rightarrow j)$ : le nouvel état obtenu en appliquant l'action $i \rightarrow j$ à l'état $E$
    \item $G = (S, A)$ : le graphe d'états, où $S$ est l'ensemble des états possibles et $A$ est l'ensemble des transitions entre les états
\end{itemize}

\subsection{Question 1}

\textit{Combien d'actions différentes sont-elles possibles pour l'état $E_1$ ?}

Supposons que l'état $E_1$ dispose de $k = 3$ piles non-vides. Chaque pile peut envoyer un bloc vers $k - 1$ autres piles. Donc, pour chaque pile, il y a $k - 1$ actions possibles. Par conséquent, le nombre total d'actions possibles pour l'état $E_1$ est $(k - 1) \times k = 6$ actions.

\subsection{Question 2}

\textit{Étant donné un état $E$ de $n$ blocs sur $k$ piles, quelle est la taille maximale de $actions(E)$ ?}

Pour maximiser le nombre d'actions possibles, chaque pile doit contenir au moins un bloc, permettant ainsi des déplacements de blocs entre toutes les piles. Ainsi, chaque pile peut envoyer un bloc à $k - 1$ autres piles. Le nombre total d'actions possibles est donc $k(k - 1)$.

\subsection{Question 3}

\textit{Étant donné un état $E$ de $n$ blocs sur $k$ piles ayant $v$ piles vides, quelle est la taille de $actions(E)$ ?}

Avec $v$ piles vides, il reste $k - v$ piles contenant des blocs. Chaque pile active peut envoyer un bloc à $k - 1$ autres piles, donc le nombre total d'actions est $(k-v)(k-1)$.

\subsection{Question 4}

\textit{Quel est l'ordre de grandeur du nombre total d'états différents possibles ?}

Le nombre total d'états est exponentiel par rapport au nombre de blocs et de piles, car chaque bloc peut être sur l'une des $k$ piles, ce qui donne une croissance exponentielle de la complexité avec l'augmentation de $n$ et $k$.

\section{Partie 2 : Définition du graphe d'états}

\subsection{Question 5}

\textit{Le graphe d'états $G$ est-il orienté ?}

Oui, le graphe d'états $G$ est orienté. Chaque action menant d'un état $E$ à un état $E'$ représente un arc orienté dans le graphe.

\subsection{Question 6}

\textit{Quels sont les algorithmes qui peuvent être utilisés pour rechercher ce plus court chemin ?}

Plusieurs algorithmes peuvent être utilisés pour rechercher le plus court chemin, tels que Bellman-Ford, BFS et Dijkstra, en fonction des spécificités du graphe (poids, orientation, etc.).

\subsection{Question 7}

\textit{Quel est l'algorithme le plus efficace pour rechercher ce plus court chemin ?}

Dans le contexte d'un graphe non pondéré comme celui du Monde des Blocs, BFS (Breadth-First Search) est généralement le plus efficace car il trouve le plus court chemin en explorant uniformément autour du nœud source.

\subsection{Question 8}

\textit{Quelle est la complexité en temps de cet algorithme par rapport à $|S|$ et $|A|$ ?}

La complexité temporelle de BFS est linéaire, soit O(|S| + |A|), où |S| est le nombre de sommets (états) et |A| est le nombre d'arêtes (actions) dans le graphe.

\subsection{Question 9}

\textit{Quelle est la complexité en temps de cet algorithme par rapport au nombre de blocs (n) et de piles (k) ?}

La complexité temporelle de l'algorithme par rapport au nombre de blocs (n) et de piles (k) est exponentielle, car le nombre total d'états possibles croît exponentiellement avec l'augmentation de n et k.

\section{Partie 3 : Heuristiques pour le monde des blocs}

\subsection{Question 10}

On commence par exécuter le programme de recherche de plus court chemin pour 4 piles et un nombre de blocs variant de 6 à 8. On obtient les résultats suivants :

\begin{minipage}{\dimexpr\linewidth-20pt}
    \begin{lstlisting}[language=bash, caption={Résultats de l'exécution du programme de recherche de plus court chemin}, label={lst:plus_court_chemin_results_no_heuristics}]
        $ make run
        Enter the number of stacks: 4
        Enter the number of blocs: 6
        Optimal solution of length 7 found in 12982 iterations and 0.060902 seconds
        [...]
        $ make run
        Enter the number of stacks: 4
        Enter the number of blocs: 7
        Optimal solution of length 9 found in 188569 iterations and 1.04769 seconds
        [...]
        $ make run
        Enter the number of stacks: 4
        Enter the number of blocs: 8
        Optimal solution of length 11 found in 2224481 iterations and 15.6446 seconds
        [...]
    \end{lstlisting}
\end{minipage}

Les résultats montrent une augmentation exponentielle du nombre d'itérations et du temps d'exécution avec l'augmentation du nombre de blocs. Pour améliorer les performances de l'algorithme, on peut adopter des heuristiques adéquates. Trois heuristiques sont introduites :

\begin{itemize}
    \item $h_1$ : Nombre de blocs ne se trouvant pas sur la dernière pile.
    \item $h_2$ : Nombre de blocs ne se trouvant pas sur la dernière pile, plus deux fois le nombre de blocs $b$ tels que $b$ se trouve sur la dernière pile mais devra nécessairement être enlevé de cette pile pour ajouter et/ou supprimer d'autres blocs sous lui.
    \item $h_3$ : Nombre de blocs de $E_i$ ne se trouvant pas sur la dernière pile, plus le nombre de blocs se trouvant au-dessus de chaque bloc ne se trouvant pas sur la dernière pile.
\end{itemize}

Pour qu'une heuristique soit admissible, elle doit toujours sous-estimer le coût réel pour atteindre l'objectif. Formellement, une heuristique $h$ est admissible si pour tout état $E$, $h(E) \leq \delta (E, E')$ où $\delta (E, E')$ est la distance réelle entre $E$ et l'état cible $E'$.

Une heuristique $h$ est dite plus informée qu'une heuristique $h'$ si, pour tout état $E$, $h(E) \geq h'(E)$, tandis que les deux heuristiques sont incomparables s'il existe deux états $E$ et $E'$ tels que $h(E) < h'(E)$ et $h(E') > h'(E')$.

\subsection{Question 11}

\textit{L'heuristique $h_1$ est-elle admissible ?}

L'heuristique $h_1$ est admissible car le nombre de blocs ne se trouvant pas sur la dernière pile représente une limite inférieure du nombre de mouvements nécessaires pour atteindre l'état final. En effet, chaque bloc doit au moins une fois être déplacé vers la dernière pile, sous-estimant ainsi le coût réel pour atteindre l'objectif.

\subsection{Question 12}

\textit{L'heuristique $h_2$ est-elle admissible ?}

L'heuristique $h_2$ est également admissible. Elle prend en compte le coût de déplacement des blocs qui sont sur la dernière pile mais doivent être déplacés pour permettre à d'autres blocs de se placer en dessous. En multipliant ce nombre par deux, on reste dans une estimation inférieure du coût réel car chaque bloc déplacé peut nécessiter plus d'un mouvement pour atteindre sa position finale.

\subsection{Question 13}

\textit{L'heuristique $h_2$ est-elle plus informée, moins informée, ou incomparable par rapport à l'heuristique $h_1$ ?}

L'heuristique $h_2$ est plus informée que l'heuristique $h_1$. En plus de compter les blocs ne se trouvant pas sur la dernière pile, elle prend en compte le coût supplémentaire pour déplacer les blocs déjà sur la dernière pile, mais qui doivent être temporairement déplacés. Ainsi, $h_2$ fournit une meilleure estimation du coût réel et est donc plus informée.

\subsection{Question 14}

\textit{L'heuristique $h_3$ est-elle admissible ?}

L'heuristique $h_3$ est admissible. Elle compte non seulement les blocs qui ne sont pas sur la dernière pile, mais aussi le nombre de blocs qui doivent être déplacés pour permettre le placement d'autres blocs. Cette mesure fournit une sous-estimation du coût réel, car elle ne prend pas en compte les déplacements multiples ni l'ordre optimal de déplacement des blocs.

\subsection{Question 15}

\textit{L'heuristique $h_3$ est-elle plus informée, moins informée, ou incomparable par rapport à l'heuristique $h_2$ ?}

L'heuristique $h_3$ est généralement moins informée que $h_2$. Tandis que $h_2$ compte de manière spécifique les déplacements nécessaires pour les blocs sur la dernière pile, $h_3$ se concentre sur une vue plus globale de tous les blocs qui ne sont pas à leur place. Cela peut conduire à des situations où $h_3$ sous-estime davantage le coût comparé à $h_2$, rendant $h_3$ moins informée dans de nombreux cas.

