% !TeX spellcheck = fr
% !TeX encoding = UTF-8

% -- Exercice 1
\section{Partie 1 : Modélisation du problème}

Dans cette partie, nous nous intéressons à la modélisation du problème du \textit{Monde des Blocs} en tant que problème de planification.

\subsection{Définition et déclaration des variables}
Voici une liste des variables utilisées, avec leur domaine et leur signification :

\begin{itemize}
    \item $n$ : le nombre de blocs
    \item $k$ : le nombre de piles
    \item $E$ : un état donné du Monde des Blocs
    \item $actions(E)$ : l'ensemble de toutes les actions possibles pour l'état $E$
    \item $t(E, i \rightarrow j)$ : le nouvel état obtenu en appliquant l'action $i \rightarrow j$ à l'état $E$
    \item $G = (S, A)$ : le graphe d'états, où $S$ est l'ensemble des états possibles et $A$ est l'ensemble des transitions entre les états
\end{itemize}

\subsection{Question 1}

\textit{Combien d'actions différentes sont-elles possibles pour l'état $E_1$ ?}

Supposons que l'état $E_1$ dispose de $k = 3$ piles non-vides. Chaque pile peut envoyer un bloc vers $k - 1$ autres piles. Donc, pour chaque pile, il y a $k - 1$ actions possibles. Par conséquent, le nombre total d'actions possibles pour l'état $E_1$ est $(k - 1) \times k = 6$ actions.

\subsection{Question 2}

\textit{Étant donné un état $E$ de $n$ blocs sur $k$ piles, quelle est la taille maximale de $actions(E)$ ?}

Pour maximiser le nombre d'actions possibles, chaque pile doit contenir au moins un bloc, permettant ainsi des déplacements de blocs entre toutes les piles. Ainsi, chaque pile peut envoyer un bloc à $k - 1$ autres piles. Le nombre total d'actions possibles est donc $k(k - 1)$.

\subsection{Question 3}

\textit{Étant donné un état $E$ de $n$ blocs sur $k$ piles ayant $v$ piles vides, quelle est la taille de $actions(E)$ ?}

Avec $v$ piles vides, il reste $k - v$ piles contenant des blocs. Chaque pile active peut envoyer un bloc à $k - 1$ autres piles, donc le nombre total d'actions est $(k-v)(k-1)$.

\subsection{Question 4}\label{q:4}

\textit{Quel est l'ordre de grandeur du nombre total d'états différents possibles ?}

Le nombre total d'états est exponentiel par rapport au nombre de blocs et linéaire par rapport au nombre de piles. 

Pour s'en convaincre, on peut considérer que pour atteindre l'état final, chaque bloc doit être déplacé au moins une fois. Considérons cette situation qui est donc une borne minimale du chemin optimal parmi le graphe d'états possibles. Dans ce cas, à chaque niveau de profondeur, on a vu avec la question 2 qu'il y a $k(k - 1)$ actions possibles au maximum. Ainsi, chaque état à chaque niveau possède $k(k - 1)$ états fils. On peut donc en déduire que le nombre total d'états est de l'ordre de $O(k(k - 1)^n) = O((k^2 - k)^n) = O(k^{2n})$.

On se retrouve alors avec une borne minimale du nombre maximum d'états possible en $O(k^2n)$, ce qui est exponentiel par rapport au nombre de blocs et linéaire par rapport au nombre de piles.

\section{Partie 2 : Définition du graphe d'états}

\subsection{Question 5}

\textit{Le graphe d'états $G$ est-il orienté ?}

La question est ambigüe. Selon le contexte réel derrière la question, le graphe d'états $G$ peut être orienté ou non. En effet, si l'on considère le graphe d'états comme un graphe de planification, alors il est orienté. En revanche, si l'on considère le graphe d'états comme un graphe de recherche de plus court chemin, alors il n'est pas orienté.

\subsection{Question 6}

\textit{Quels sont les algorithmes qui peuvent être utilisés pour rechercher ce plus court chemin ?}

Plusieurs algorithmes peuvent être utilisés pour rechercher le plus court chemin, tels que Bellman-Ford, BFS, DFS et Dijkstra, en fonction des spécificités du graphe (poids, orientation, etc.).

Attention, on ne pourra pas utiliser l'algorithme TopoDAG car le graphe n'est pas un DAG (Directed Acyclic Graph) car il contient des cycles. En effet, par exemple, l'état final peut typiquement être atteint par plusieurs états prédédecesseurs différents.

\subsection{Question 7}

\textit{Quel est l'algorithme le plus efficace pour rechercher ce plus court chemin ?}

Dans le contexte d'un graphe non pondéré comme celui du Monde des Blocs, BFS (Breadth-First Search) est généralement le plus efficace, car il trouve le plus court chemin en explorant uniformément autour du nœud source.

DFS est un autre algorithme de parcours de graphe qui se concentre sur l'exploration en profondeur d'une branche du graphe avant de revenir en arrière pour explorer d'autres branches. Il peut également être utilisé pour trouver le plus court chemin. 

Chacun de ces algorithmes a ses avantages et inconvénients. Mais surtout, ils peuvent être améliorés en utilisant des heuristiques pour guider la recherche vers des solutions plus optimales.

\subsection{Question 8}

\textit{Quelle est la complexité en temps de cet algorithme par rapport à $|S|$ et $|A|$ ?}

La complexité temporelle de BFS est linéaire, soit O(|S| + |A|), où |S| est le nombre de sommets (états) et |A| est le nombre d'arêtes (actions) dans le graphe.

\subsection{Question 9}

\textit{Quelle est la complexité en temps de cet algorithme par rapport au nombre de blocs (n) et de piles (k) ?}

La complexité temporelle de l'algorithme par rapport au nombre de blocs (n) et de piles (k) est exponentielle, car le nombre total d'états possibles croît exponentiellement avec l'augmentation de n et k (voir question 4, section \ref{q:4}).

\section{Partie 3 : Heuristiques pour le monde des blocs}

\subsection{Question 10}

On commence par exécuter le programme de recherche de plus court chemin pour 4 piles et un nombre de blocs variant de 6 à 8. On obtient les résultats suivants :

\begin{minipage}{\dimexpr\linewidth-20pt}
    \begin{lstlisting}[language=bash, caption={Résultats de l'exécution du programme de recherche de plus court chemin}, label={lst:plus_court_chemin_results_no_heuristics}]
        $ make run
        Enter the number of stacks: 4
        Enter the number of blocs: 6
        Optimal solution of length 7 found in 12982 iterations and 0.060902 seconds
        [...]
        $ make run
        Enter the number of stacks: 4
        Enter the number of blocs: 7
        Optimal solution of length 9 found in 188569 iterations and 1.04769 seconds
        [...]
        $ make run
        Enter the number of stacks: 4
        Enter the number of blocs: 8
        Optimal solution of length 11 found in 2224481 iterations and 15.6446 seconds
        [...]
    \end{lstlisting}
\end{minipage}

Les résultats montrent une augmentation exponentielle du nombre d'itérations et du temps d'exécution avec l'augmentation du nombre de blocs. Pour améliorer les performances de l'algorithme, on peut adopter des heuristiques adéquates. Trois heuristiques sont introduites :

\begin{itemize}
    \item $h_1$ : Nombre de blocs ne se trouvant pas sur la dernière pile.
    \item $h_2$ : Nombre de blocs ne se trouvant pas sur la dernière pile, plus deux fois le nombre de blocs $b$ tels que $b$ se trouve sur la dernière pile mais devra nécessairement être enlevé de cette pile pour ajouter et/ou supprimer d'autres blocs sous lui.
    \item $h_3$ : Nombre de blocs de $E_i$ ne se trouvant pas sur la dernière pile, plus le nombre de blocs se trouvant au-dessus de chaque bloc ne se trouvant pas sur la dernière pile.
\end{itemize}

Pour qu'une heuristique soit admissible, elle doit toujours sous-estimer le coût réel pour atteindre l'objectif. Formellement, une heuristique $h$ est admissible si pour tout état $E$, $h(E) \leq \delta (E, E')$ où $\delta (E, E')$ est la distance réelle entre $E$ et l'état cible $E'$.

Une heuristique $h$ est dite plus informée qu'une heuristique $h'$ si, pour tout état $E$, $h(E) \geq h'(E)$, tandis que les deux heuristiques sont incomparables s'il existe deux états $E$ et $E'$ tels que $h(E) < h'(E)$ et $h(E') > h'(E')$.

\subsection{Question 11}

\textit{L'heuristique $h_1$ est-elle admissible ?}

L'heuristique $h_1$ est admissible, car le nombre de blocs ne se trouvant pas sur la dernière pile représente une limite inférieure du nombre de mouvements nécessaires pour atteindre l'état final. En effet, chaque bloc doit au moins une fois être déplacé vers la dernière pile, sous-estimant ainsi le coût réel pour atteindre l'objectif.

\subsection{Question 12}

\textit{L'heuristique $h_2$ est-elle admissible ?}

L'heuristique $h_2$ est également admissible. Elle prend en compte le coût de déplacement des blocs qui sont sur la dernière pile, mais doivent être déplacés pour permettre à d'autres blocs de se placer en dessous. En multipliant ce nombre par deux, on reste dans une estimation inférieure du coût réel. En effet, pour chaque bloc sur la dernière pile, mais n'étant pas à la bonne hauteur, il faudra au minimum 2 déplacements de ce bloc considéré pour l'enlever puis le remettre sur la dernière pile. Il s'agit d'une estimation minimum, car chaque bloc ainsi déplacé peut nécessiter plus de 2 mouvements pour atteindre sa position finale.

\subsection{Question 13}

\textit{L'heuristique $h_2$ est-elle plus informée, moins informée, ou incomparable par rapport à l'heuristique $h_1$ ?}

L'heuristique $h_2$ est plus informée que l'heuristique $h_1$. En plus de compter les blocs ne se trouvant pas sur la dernière pile, elle prend en compte le coût supplémentaire pour déplacer les blocs déjà sur la dernière pile, mais qui doivent être temporairement déplacés. Ainsi, $h_2$ donnera une valeur de coût égale ou supérieur à celle fournie par $h_1$, et donne donc une meilleure estimation du coût réel. Elle est donc plus informée.

\subsection{Question 14}

\textit{L'heuristique $h_3$ est-elle admissible ?}

L'heuristique $h_3$ n'est PAS admissible. Elle compte non seulement les blocs qui ne sont pas sur la dernière pile, mais aussi le nombre de blocs qui doivent être déplacés pour permettre le placement d'autres blocs. Il existe des cas dans lesquels $h_3$ fournit une surestimation du coût réel, car elle ne prend pas en compte les déplacements multiples ni l'ordre optimal de déplacement des blocs.

Un contre exemple simple consiste à considérer un exemple avec 2 piles et 2 blocs sur la pile 0, de tel sorte que ces blocs peuvent être placés directement sur la pile 1. Dans ce cas, $h_3$ donnera une valeur de $2 + 1 = 3$ (2 blocs pas sur la dernière pile plus le bloc du dessous qui a un bloc au dessus de lui), alors que le coût réel est de 2. 

\subsection{Question 15}

\textit{L'heuristique $h_3$ est-elle plus informée, moins informée, ou incomparable par rapport à l'heuristique $h_2$ ?}

Tandis que $h_2$ compte de manière spécifique les déplacements nécessaires pour les blocs sur la dernière pile, $h_3$ se concentre sur une vue plus globale de tous les blocs qui ne sont pas à leur place. En fonction de la situation, par exemple si la dernière pile est déjà bien remplie, $h_2$ peut donner une meilleure estimation du coût réel que $h_3$ puisqu'il reste peu de blocs sur les autres piles. C'est bien sûr l'inverse quand la dernière pile est vide. Ainsi, les deux heuristiques sont incomparables.

On peut ainsi trouver des contres-exemples :

\[
    \exists E_1, E_2 \in E \ | \ h_2(E_1) < h_2(E_2) \land h_3(E_1) > h_3(E_2).
\]

\begin{figure}
    \centering

    \begin{tabular}{ccc} 
        \scalebox{0.6}{
        \begin{tikzpicture}
            \tikzstyle{every node}=[font=\LARGE]
            \draw (4,24.75) rectangle node {\LARGE a} (5.75,23);
            \draw (4,22.75) rectangle node {\LARGE d} (5.75,21);
            \draw (7,22.75) rectangle node {\LARGE c} (8.75,21);
            \draw (10,22.75) rectangle node {\LARGE e} (11.75,21);
            \draw (10,24.75) rectangle node {\LARGE b} (11.75,23);
            \draw [](4,20.75) to (5.75,20.75);
            \draw [](7,20.75) to (8.75,20.75);
            \draw [](10,20.75) to (11.75,20.75);
            \node [font=\LARGE] at (5,20) {0};
            \node [font=\LARGE] at (8,20) {1};
            \node [font=\LARGE] at (11,20) {2};
            \node [font=\LARGE] at (8,19) {$E_1$};
        \end{tikzpicture}
        }
        
        & \hspace{2cm} &

        \scalebox{0.6}{
        \begin{tikzpicture}
            \tikzstyle{every node}=[font=\LARGE]
            \draw (4,24.75) rectangle node {\LARGE a} (5.75,23);
            \draw (4,22.75) rectangle node {\LARGE d} (5.75,21);
            \draw (7,22.75) rectangle node {\LARGE c} (8.75,21);
            \draw (10,22.75) rectangle node {\LARGE e} (11.75,21);
            \draw (7,24.75) rectangle node {\LARGE b} (8.75,23);
            \draw [](4,20.75) to (5.75,20.75);
            \draw [](7,20.75) to (8.75,20.75);
            \draw [](10,20.75) to (11.75,20.75);
            \node [font=\LARGE] at (5,20) {0};
            \node [font=\LARGE] at (8,20) {1};
            \node [font=\LARGE] at (11,20) {2};
            \node [font=\LARGE] at (8,19) {$E_2$};
        \end{tikzpicture}
        }
    \end{tabular}
    
    \caption{Exemples de deux états $E_1$ et $E_2$, pour $k = 3$ et $n = 5$.}.
    \label{fig:contre_exemple_h2_h3}
\end{figure}

Dans l'illustration suivante \ref{fig:contre_exemple_h2_h3}, on peut voir :

\[
    h(E_1) = 
    \begin{cases} 
        h_1(E_1) = 3 \\
        h_2(E_1) = 5 \\
        h_3(E_1) = 4
    \end{cases}
    \quad
    h(E_2) =
    \begin{cases} 
        h_1(E_2) = 4 \\
        h_2(E_2) = 4 \\
        h_3(E_2) = 6
    \end{cases}
\]

On a donc $h_2(E_1) > h_3(E_2) \land h_2(E_1) < h_3(E_2)$, ce qui montre que les deux heuristiques sont incomparables.

\subsection{Question 16 : heuristique $h_4$}

\textit{Quelle est la longueur (en nombre d'actions) de la solution optimale pour $n = 16$ blocs et $k = 4 piles$ ?}

On commence par implémenter les trois heuristiques dans le programme de recherche de plus court chemin. On obtient nos premiers résultats :

\begin{minipage}{\dimexpr\linewidth-20pt}
    \begin{lstlisting}[language=bash, caption={Résultats de l'exécution du programme de recherche de plus court chemin avec 3 premières heuristiques pour $k = 4$ piles et $n = 8$ blocs.}, label={lst:plus_court_chemin_results_h1-3_first_res}]
        $ make run
        [...]
        g(4, 8, heuristic0): 
        Optimal solution of length 11 found in 2224481 iterations and 15.1703 seconds
        g(4, 8, heuristic1): 
        Optimal solution of length 11 found in 61468 iterations and 0.242329 seconds
        g(4, 8, heuristic2): 
        Optimal solution of length 11 found in 169 iterations and 0.000934 seconds
        g(4, 8, heuristic3): 
        Optimal solution of length 12 found in 261465 iterations and 1.12496 seconds
    \end{lstlisting}
\end{minipage}

On remarque d'après les résultats obtenus \ref{lst:plus_court_chemin_results_h1-3_first_res} que l'heuristique $h_2$ est la plus performante, car elle trouve la solution optimale en un nombre d'itérations et de temps d'exécution très faible. On peut donc la considérer comme la meilleure heuristique parmi les trois.

On ajoute alors une quatrième heuristique $h_4$ basée sur $h_2$, elle-même une amélioration de $h_1$. Pour cela, considérons par exemples les états $E_1$ et $E_2$ suivants :

\begin{figure}[!ht]
    \centering

    \begin{tabular}{ccc} 
        \scalebox{0.6}{
        \begin{tikzpicture}
            \tikzstyle{every node}=[font=\LARGE]
            \draw (4,24.75) rectangle node {\LARGE 3} (5.75,23);
            \draw (4,22.75) rectangle node {\LARGE 4} (5.75,21);
            \draw (7,22.75) rectangle node {\LARGE 2} (8.75,21);
            \draw (10,22.75) rectangle node {\LARGE 0} (11.75,21);
            \draw (7,24.75) rectangle node {\LARGE 1} (8.75,23);
            \draw [](4,20.75) to (5.75,20.75);
            \draw [](7,20.75) to (8.75,20.75);
            \draw [](10,20.75) to (11.75,20.75);
            \node [font=\LARGE] at (5,20) {0};
            \node [font=\LARGE] at (8,20) {1};
            \node [font=\LARGE] at (11,20) {2};
            \node [font=\LARGE] at (8,19) {$E_1$};
        \end{tikzpicture}
        }
        
        & \hspace{2cm} &

        \scalebox{0.6}{
        \begin{tikzpicture}
            \tikzstyle{every node}=[font=\LARGE]
            \draw (4,24.75) rectangle node {\LARGE 4} (5.75,23);
            \draw (4,22.75) rectangle node {\LARGE 3} (5.75,21);
            \draw (7,22.75) rectangle node {\LARGE 1} (8.75,21);
            \draw (10,22.75) rectangle node {\LARGE 0} (11.75,21);
            \draw (7,24.75) rectangle node {\LARGE 3} (8.75,23);
            \draw [](4,20.75) to (5.75,20.75);
            \draw [](7,20.75) to (8.75,20.75);
            \draw [](10,20.75) to (11.75,20.75);
            \node [font=\LARGE] at (5,20) {0};
            \node [font=\LARGE] at (8,20) {1};
            \node [font=\LARGE] at (11,20) {2};
            \node [font=\LARGE] at (8,19) {$E_2$};
        \end{tikzpicture}
        }
    \end{tabular}
    
    \caption{Exemples de deux états $E_1$ et $E_2$, pour $k = 3$ et $n = 5$, illustrant l'intérêt derrière $h_4$.}
    \label{fig:exemple_pour_h4}
\end{figure}

On peut constater à la main que l'état $E_2$ est à 7 mouvements de l'état cible, alors que l'état $E_1$ en a besoin de beaucoup plus.

On définit alors une telle heuristique $h_4$ comme suit :

\begin{itemize}
    \item \textbf{Nombre de blocs non dans la dernière pile :} Correspondant à \(h_1\), cette partie compte tous les blocs qui ne sont pas placés dans la dernière pile. Plus il y a de blocs éloignés de leur destination finale, plus le nombre de mouvements requis sera potentiellement important.
    
    \item \textbf{Deux fois le nombre de blocs sur la dernière pile à déplacer :} Cette mesure reprends \(h_2\) en donnant plus de poids aux blocs qui doivent être déplacés depuis la dernière pile. Cela reflète le coût potentiellement élevé en termes de mouvements pour déplacer ces blocs, notamment s'il faut les déplacer temporairement ailleurs avant de les remettre en ordre.
    
    \item \textbf{Évaluation de l'arrangement des blocs sur toutes les autres piles :} Pour chaque pile (sauf la dernière), 
    on augmente la somme heuristique pour chaque bloc qui n'est pas plus grand que le bloc situé en dessous. Ainsi, un bloc plus petit ou de taille égale reposant sur un bloc plus grand augmente la valeur heuristique, indiquant un état moins idéal. Cela capture l'arrangement des blocs dans les piles et favorise les états où ils sont déjà dans un ordre croissant vers le but.
\end{itemize}

En combinant ces éléments, l'heuristique $h_4$ fournit une meilleure sous-estimation que $h_1$ ou $h_2$, en saisissant davantage de nuances dans l'agencement des blocs. Cette sous-estimation est importante car elle garantit que l'heuristique reste admissible et ne surestime jamais le coût réel pour atteindre l'objectif, ce qui est une propriété essentielle pour guarantir l'optimalité de A*.

On obtient les résultats suivants pour cette nouvelle heuristique :

\begin{minipage}{\dimexpr\linewidth-20pt}
    \begin{lstlisting}[language=bash, caption={Résultats de l'exécution du programme de recherche de plus court chemin avec l'heuristique $h_4$ pour $k = 4$ piles et $n = 8$ blocs.}, label={lst:plus_court_chemin_results_h4_first_res}]
        $ make run
        [...]
        g(4, 8, heuristic4): 
        Optimal solution of length 11 found in 50 iterations and 0.000341 seconds
    \end{lstlisting}
\end{minipage}

Comme on peut s'y attendre, l'heuristique $h_4$ est plus performante que les trois premières heuristiques. Elle trouve la solution optimale en un nombre d'itérations et de temps d'exécution très faible. On peut donc la considérer comme la meilleure heuristique parmi les quatre. C'est cette heuristique qui sera utilisée pour la suite des expériences.

\subsection{Question 17}

\textit{Quelle est la longueur (en nombre d'actions) de la solution optimale pour $n = 20$ blocs et $k = 4$ piles ?}

On exécute le programme de recherche de plus court chemin avec l'heuristique $h_4$ pour $n = 20$ blocs et $k = 4$ piles. On obtient les résultats suivants :

\begin{minipage}{\dimexpr\linewidth-20pt}
    \begin{lstlisting}[language=bash, caption={Résultats de l'exécution du programme de recherche de plus court chemin avec l'heuristique $h_4$ pour $k = 4$ piles et $n = 20$ blocs.}, label={lst:plus_court_chemin_results_h4_k4_n20}]
        $ make run
        [...]
        g(4, 20, heuristic4): 
        Optimal solution of length 32 found in 306805 iterations and 3.11438 seconds
    \end{lstlisting}
\end{minipage}

La longueur de la solution optimale est de 32 actions.

\subsection{Question 18}

\textit{Quelle est la longueur (en nombre d'actions) de la solution optimale pour $n = 25$ blocs et $k = 5$ piles ?}

Afin de trouver la longueur de la solution optimale pour autant de calculs nécessaires, on transforme le programme de recherche de plus court chemin en passant de \textit{A*} à \textit{AWA*} (Anytime Weighted A*). AWA* est une variante de A* qui permet de trouver une solution optimale en un temps limité. Il s'agit d'une version pondérée de A*, où le poids $w$ est défini par l'utilisateur. Ainsi, AWA* peut être utilisé pour trouver une solution optimale en un temps limité, ou pour trouver une solution sous-optimale en un temps limité.

\begin{itemize}
    \item \textbf{Algorithme WA* (Weighted A*) :} Version très légèrement modifiée de l'algorithme A*, où la valeur de mise en file d'attente dans le tas est davantage pondérée en faveur de l'heuristique. Ceci transforme techniquement A* en une sorte de Recherche Gloutonne au Meilleur Premier (Greedy Best First Search). Ainsi, lorsque \(w = 1\), la distance depuis l'origine et la distance heuristique vers le but sont pondérées de manière égale et l'algorithme se comporte exactement comme A*. Lorsque \(w > 1\), l'algorithme trouve une solution plus rapidement mais perd la garantie d'optimalité. L'objectif est d'itérer vers un \(w = 1\).
    
    \item \textbf{Algorithme AWA* (Anytime Weighted A*) :} Variante améliorée et anytime de l'algorithme WA*. L'objectif ici est de mettre à jour une borne supérieure de la solution. Cette approche permet à l'algorithme de fournir une solution valide à tout moment du processus de recherche et d'améliorer cette solution au fur et à mesure du temps ou des itérations disponibles.
\end{itemize}

Après implémentation de AWA*, et en utilisant $h_4$, on obtient les résultats suivants :

\begin{minipage}{\dimexpr\linewidth-20pt}
    \begin{lstlisting}[language=bash, caption={Résultats de l'exécution du programme de recherche de plus court chemin avec l'heuristique $h_4$ et \textit{AWA*} pour $k = 5$ piles et $n = 25$ blocs.}, label={lst:plus_court_chemin_results_h4_k4_n20}]
        $ make run
        [...]
        g(5, 25, heuristic4): 
        starting upper bound = 2147483647, weight = 2
        Found solution of length 42 in 1539 iterations and 0.032113 seconds
        Found solution of length 41 in 1678 iterations and 0.03426 seconds
        Found solution of length 40 in 1774 iterations and 0.03594 seconds
        Found solution of length 39 in 8472 iterations and 0.149428 seconds
        Found solution of length 38 in 11642 iterations and 0.210004 seconds
        Found solution of length 37 in 19889 iterations and 0.374129 seconds
        Optimal solution of length 37 found in 1066478 iterations and 21.8191 seconds
    \end{lstlisting}
\end{minipage}

On peut voir que la longueur de la solution optimale est de 37 actions.

